\chapter{Fundamentação Teórica}

\lipsum[1-2]

\begin{citacao}
\lipsum[1]\cite[p. ~34]{Huizinga2014}
\end{citacao}

\section{O que é Inteligência Artifical}


\lipsum[4-7]

\section{Processos Já automatizados com IA na área de TI}

\lipsum[8-9]

A figura \ref{fig:kings-landing} representa \emph{King's Landing}, cenário da série \emph{Game of Thromes} reproduzido no Minecraft:

\begin{figure}[h]
	\caption{\emph{The King's Landing} no Minecraft}
	\center
	\label{fig:kings-landing}
	\includegraphics[scale=0.15]{fundamentacao/kings-landing.jpg}
	\fol{Imgur2013}
\end{figure}

\lipsum[10-12]

\section{Papel atual de um DBA}

DBA ou DataBase Administrator é a individuo responsável pelo gerenciamento da infraestrutura do banco de dados, ou seja, recursos ali disponíveis de uma determinada organização, tendo o papel considerado vital na mesma, visto que é responsável por toda informação contida na base de dados.


Podemos dizer que o DBA cuida de certa forma do coração da organização, onde são responsáveis por:
\begin{alineas}
\item Avaliação e definição de Hardware necessário para instalação do Banco de Dados;
\item Instalação e gerenciamento do banco de dados que envolve toda parte de software, atualizações e correções de bugs;
\item Planejamento e implementação do banco de dados, definindo e criando as tabelas, índices e outros objetos de banco;
\item Criação de rotinas de backups, bem com garantir que os mesmos estejam disponíveis para recuperação.
\item Criação de estratégias para plano de recuperação de desastres;
\item Gerenciamento de usuários, realizando a criação e liberando os privilégios mínimos necessários para exercer suas atividades;
\item Monitoramento e garantia de performance do ambiente, onde utiliza-se de várias ferramentas para manter o ambiente sempre ativo e com um tempo de resposta aceitável pelo usuário.
\end{alineas} 


Diante as atividades citadas anteriormente uma das tarefas mais complexas a serem realizadas pelo DBA é a necessidade de uma alteração na base de dados seja na estrutura lógica de uma determinada tabela ou na criação de uma nova, pois o mesmo deve conhecer dos processos de modelagem de dados para identificar possíveis implicações e/ou como resolve-las.


Logo podemos dizer que nos dias atuais o DBA tem um papel fundamental para as organizações no processo de automatizações de processos relacionados as informações armazenadas no banco de dados.


\section{O que é DB autônomo}

\begin{citacao}
Em 2017, a Oracle anunciou o primeiro banco de dados autônomo do mundo. Com a novidade, o administrador de banco de dados (DBA) passa a se concentrar em outras tarefas que agregam valor à empresa, em vez de investir tempo com rotinas que poderiam ser feitas pelas próprias máquinas, com maior eficiência e sem intervenção humana.\cite[p. ~34]{TecnoBlog}
\end{citacao}

Então o que seria um banco de dados autônomo? É a automatização dos processos e rotinas do banco de dados sem a intervenção do papel humano. Ou seja, o banco de dados fica localizado na nuvem que usa machine learning, onde toda otimização, reparação e instalação de patches de segurança, deixam de existir no papel de DBA, pois todo processo é feito pelo próprio database.


Ainda de acordo com a Oracle, \cite{OracleDataBase}, um banco de dados autônomo tem quatros metas abrangentes:
\begin{alineas}
\item Garantir o máximo de tempo de atividade e desempenho do banco de dados;
\item Garantir a segurança máxima do banco de dados, incluindo patches e correções;
\item Eliminar tarefas de gerenciamento manuais e propensas a erros com a automação;
\item Permitir que os DBAs apliquem seus conhecimentos a funções de nível superior.
\end{alineas}


Logo para garantir as metas citadas anteriormente podemos descrever os recursos principais da seguinte forma:

\begin{alineas}
\item Autoexecução: realiza otimizações de desempenho de forma continua, como processos de cache automático, indexado adaptável e compressão avançada. Ele também ajusta de forma instantânea os recursos de processamento e armazenamentos de dados
\item Autoproteção: realiza a instalação e atualizações de segurança sem a necessidade da intervenção humana e parada do banco de dados. Faz a proteção dos dados da organização contra ataques cibernético tanto interno, quanto externos
\item Autorreparação: realiza a proteção autônoma contra as paradas inesperadas.
\end{alineas}


Por mais eficiente que seja o profissional responsável pela administração do banco de dados, a tecnologia autônoma com inteligência artificial e machine learning (Aprendizado de Máquina) que é um método de análise de dados para que as maquinas possam aprender com os conjuntos de informações absorvidas a melhor forma de otimizar e automatizar processos/rotinas, reduzindo assim drasticamente as possibilidades de erros.


A tecnologia de banco de dado autônomos ainda é muito nova, porém algumas organizações já começaram a utilizar da mesma como por exemplo a SKY Brasil que em 2019 aderiu o Oracle Autonomous Database, onde o diretor de TI André Nazaré da empresa afirma que "Além do alto desempenho, a adoção do Oracle Autonomous Database simplificou os processos e permitiu chegarmos aos nossos clientes com a oferta certa. A SKY espera usar ainda mais a solução e levar toda sua infraestrutura para a nuvem. Em menos de um ano do início da migração para a nuvem Oracle, a operadora economizou US 750 mil em infraestrutura e operações on-premise de seu banco de dados" . \cite{Sky}

Portanto o banco de dados autônomo veio para ajudar a área de TI a manter uma ambiente mais confiável e seguro com relação aos dados das organizações, além de proporcionar uma redução de custos e aproveitar dos DBAs em papeis para estratégicos dentro da empresa.



\section{Como funciona um autônomo}

\lipsum[1-2]

\lipsum[10-12]
\lipsum[10-12]

\section{Quais os recursos que é possível automatizar do BD}

\lipsum[1-2]

\lipsum[10-12]
\lipsum[10-12]
\lipsum[10-12]




\section{Qual o novo papel do DBA}

\lipsum[1-2]


\section{Ferramentas que existem e o que fazem}

\lipsum[1-20]

