\chapter{Fundamentação Teórica}
\section{O que é Inteligência Artifical}
É um ramo da ciência da computação que propõe elaborar dispositivos ou mecanismos que simules uma capacidade humana, ou seja, com uma certa entrada de dados ou conhecimentos, esta IA possa processar e aprender novas alternativas automaticamente sem a intervenção de terceiros ou da mão humana, aprendendo sozinha e decidindo a melhor forma de resolver um problema ou aperfeiçoar procedimentos já existentes. 

\section{Processos Já automatizados com IA na área de TI}
	Em pleno século XXI temos vários exemplos de IA em desenvolvimento em enumeras áreas. Assim podemos colocar como exemplo os carros autônomos, onde podem ir de um ponto A ao ponto B, escolhendo o melhor percurso sendo pelo menor tempo, pelo menor tráfego, programando automaticamente a sua hora de abastecimento, e tomando decisões para desviar de obstáculos.
	Assim como o setor automobilístico, temos vários outros seguindo a mesma tendência, como a aviação, a medicina, setores financeiros e o setor de TI (tecnologia da informação), sempre visando aperfeiçoar os processos sem a intervenção humana, deixando a mão de obra livre para melhorar ou desenvolver outros projetos. Exemplos populares de IA, SIRI ( Apple), cortana ( microsoft ) e o google assistant. 
	
	Com crescimento do poder computacional, e os avanços da tecnologia como Cloud Computing e Big data esta atividade vem ganhando autonomia cada vez maior. Com o grande parque de banco de dados e aglomerado de servidores, seria quase impossível ou inviável gerenciar equipes para manutenção e ajustes, mas com algoritmos e protocolos que façam o uso da IA e Machine learning ( aprendizado da máquina ), isso se torna muito viável com custo muito baixo.
	Entrando um pouco mais no mundo da TI, a análise massiva de dados  foi um dos primeiros processos, a serem automatizados pela IA. A demanda no volume de dados, não só o acesso como o armazenamento e organização dos dados tiveram que ser aperfeiçoados e automatizados, um exemplo simples que passa despercebido  é o SPAM dos eu e-mail, através de uma analise comportamental o algoritmo analisa e decide que é ou não liberado para a caixa de entrada ou será enviado para  o spam. 


\section{Papel atual de um DBA}

DBA ou DataBase Administrator é a individuo responsável pelo gerenciamento da infraestrutura do banco de dados, ou seja, recursos ali disponíveis de uma determinada organização, tendo o papel considerado vital na mesma, visto que é responsável por toda informação contida na base de dados.


Podemos dizer que o DBA cuida de certa forma do coração da organização, onde são responsáveis por:
\begin{alineas}
\item Avaliação e definição de Hardware necessário para instalação do Banco de Dados;
\item Instalação e gerenciamento do banco de dados que envolve toda parte de software, atualizações e correções de bugs;
\item Planejamento e implementação do banco de dados, definindo e criando as tabelas, índices e outros objetos de banco;
\item Criação de rotinas de backups, bem com garantir que os mesmos estejam disponíveis para recuperação.
\item Criação de estratégias para plano de recuperação de desastres;
\item Gerenciamento de usuários, realizando a criação e liberando os privilégios mínimos necessários para exercer suas atividades;
\item Monitoramento e garantia de performance do ambiente, onde utiliza-se de várias ferramentas para manter o ambiente sempre ativo e com um tempo de resposta aceitável pelo usuário.
\end{alineas} 


Diante as atividades citadas anteriormente uma das tarefas mais complexas a serem realizadas pelo DBA é a necessidade de uma alteração na base de dados seja na estrutura lógica de uma determinada tabela ou na criação de uma nova, pois o mesmo deve conhecer dos processos de modelagem de dados para identificar possíveis implicações e/ou como resolve-las.


Logo podemos dizer que nos dias atuais o DBA tem um papel fundamental para as organizações no processo de automatizações de processos relacionados as informações armazenadas no banco de dados.


\section{O que é DB autônomo}

\begin{citacao}
Em 2017, a Oracle anunciou o primeiro banco de dados autônomo do mundo. Com a novidade, o administrador de banco de dados (DBA) passa a se concentrar em outras tarefas que agregam valor à empresa, em vez de investir tempo com rotinas que poderiam ser feitas pelas próprias máquinas, com maior eficiência e sem intervenção humana.\cite[p. ~34]{TecnoBlog}
\end{citacao}

Então o que seria um banco de dados autônomo? É a automatização dos processos e rotinas do banco de dados sem a intervenção do papel humano. Ou seja, o banco de dados fica localizado na nuvem que usa machine learning, onde toda otimização, reparação e instalação de patches de segurança, deixam de existir no papel de DBA, pois todo processo é feito pelo próprio database.


Ainda de acordo com a Oracle, \cite{OracleDataBase}, um banco de dados autônomo tem quatros metas abrangentes:
\begin{alineas}
\item Garantir o máximo de tempo de atividade e desempenho do banco de dados;
\item Garantir a segurança máxima do banco de dados, incluindo patches e correções;
\item Eliminar tarefas de gerenciamento manuais e propensas a erros com a automação;
\item Permitir que os DBAs apliquem seus conhecimentos a funções de nível superior.
\end{alineas}


Logo para garantir as metas citadas anteriormente podemos descrever os recursos principais da seguinte forma:

\begin{alineas}
\item Autoexecução: realiza otimizações de desempenho de forma continua, como processos de cache automático, indexado adaptável e compressão avançada. Ele também ajusta de forma instantânea os recursos de processamento e armazenamentos de dados
\item Autoproteção: realiza a instalação e atualizações de segurança sem a necessidade da intervenção humana e parada do banco de dados. Faz a proteção dos dados da organização contra ataques cibernético tanto interno, quanto externos
\item Autorreparação: realiza a proteção autônoma contra as paradas inesperadas.
\end{alineas}


Por mais eficiente que seja o profissional responsável pela administração do banco de dados, a tecnologia autônoma com inteligência artificial e machine learning (Aprendizado de Máquina) que é um método de análise de dados para que as maquinas possam aprender com os conjuntos de informações absorvidas a melhor forma de otimizar e automatizar processos/rotinas, reduzindo assim drasticamente as possibilidades de erros.


A tecnologia de banco de dado autônomos ainda é muito nova, porém algumas organizações já começaram a utilizar da mesma como por exemplo a SKY Brasil que em 2019 aderiu o Oracle Autonomous Database, onde o diretor de TI André Nazaré da empresa afirma que "Além do alto desempenho, a adoção do Oracle Autonomous Database simplificou os processos e permitiu chegarmos aos nossos clientes com a oferta certa. A SKY espera usar ainda mais a solução e levar toda sua infraestrutura para a nuvem. Em menos de um ano do início da migração para a nuvem Oracle, a operadora economizou US 750 mil em infraestrutura e operações on-premise de seu banco de dados" . \cite{Sky}

Portanto o banco de dados autônomo veio para ajudar a área de TI a manter uma ambiente mais confiável e seguro com relação aos dados das organizações, além de proporcionar uma redução de custos e aproveitar dos DBAs em papeis para estratégicos dentro da empresa.



\section{Como funciona um autônomo}

O banco de dados autônomo é capaz de automatizar de forma inteligente as principais funções do DBA que são Otimização de Desempenho, gerenciamento de cache, atualização do SGBD, Backup, Segurança de Dados, restauração em casos de desastres entre outras, mesmo tendo um ótimo profissional erros sempre vão existir e essas tarefas sendo executadas pela máquina significa uma redução drástica na porcentagem de erros. Além disso, essa tecnologia é capaz de aumentar o desempenho e a disponibilidade do SGBD, sendo essas funções divididas em 3 pilares \cite{OracleDBAutonomo}:


	Autuexecução: Tendo seu fundamento principal a machine learning, onde o sistema tem a capacidade de aprender com suas decisões, sendo possível de forma autônoma a realização de melhoria de desempenho de forma ininterrupta, onde existe manipulação do cache onde que de forma inteligente aloca dinamicamente os dados facilitando consultas e outras necessidades, tem indexação de índices de forma autônoma e adaptável a cada necessidade criando e apagando índices para dar agilidade a aplicação que utiliza os dados do banco de dados onde o usuário ou desenvolvedor só precisa se preocupara em carregar e utilizar os dados e o SGBD fica responsável em criar e gerenciar as estruturas de acessos aos dados solicitados, de forma transparente ele tem o poder de gerenciar o armazenamento do banco comprimindo e descomprimindo colunas de uma tabela, partição ou Tablespace com isso é possível ter mais informações utilizando o menor espaço possível, um banco de dados de 900 GB pode reduzir para 240 GB sem perder desempenho \cite{OracleExadata}, também é possível de forma automática alocar e desalocar recursos de processador, memória RAM e disco rígido, onde com o monitoramento automático o sistema informa ao gerenciadores dos servidores se há a necessidade de aumentar ou diminuir esses recursos, no mundo onde a virtualização é algo muito comum nas medias e grandes empresas onde com essa tecnologia é possível, sem desligar um servidor virtualizado, aumentar e diminuir processador, memória RAM e disco de armazenamento, por exemplo: um servidor virtualizado tem 6 vCPU’s, 8 GB de Memória RAM e 60 GB de armazenamento e se faz necessário adicionar ou remover recursos, o sistema de gerenciamento de banco de dados autônomo é capaz de mudar as suas configurações de alocação de recurso aumentando ou diminuindo sem a necessidade de reiniciar ou parar o servidor, caso haja usuários conectados ao banco de dados os mesmo permanecem e nem percebem as mudanças realizadas, algo muito parecido com a tecnologia hot swap (Troca a quente) muito utilizada em discos de armazenamento ou Hot-add muito utilizada em ambientes virtualizados para adicionar processador, memória RAM e disco de armazenamentos para maquinas virtuais sem precisar para nenhuma máquina \cite{OracleBreakingFree}.
	
	
	Autoproteção: podemos observar que nos últimos anos os ataques cibernéticos tem crescido de forma exponencial muitos deles poderiam ser evitados com a aplicação de um simples pacote de atualização para corrigir as vulnerabilidades o problemas deles é que na maioria das instalações se faz necessário a suspenção do serviço do banco de dados pelo período de atualização, sendo que aconteça alguma falha na aplicação do pacote será necessário remover o pacote que deu erro ou refazer a instalação do sistema de gerenciamento do banco de dados, já com o banco de dados autônomo ele consegue fazer isso forma transparente para o usuário sem parar o serviço do banco de dados e sem a interferência do administrador de banco de dados, pois ele cria uma cópia do sistema atual, mantendo a atual em produção, depois atualiza e aplica todos os pacotes de atualizações e segurança testando o sistema novo, após os testes sincroniza os dados e o novo sistema totalmente atualizado e funcional assume o papel principal, descantando o anterior. Outra falha que gera vários ataques é a falta de criptografia no banco de dados por questão de desempenho ou até falta de conhecimento, de formar padrão todo banco de dados autônomo aplica criptografia de 2 camadas no banco de dados, nível de banco de dados e nível de aplicação, no banco de dados é possível criptografar o banco completo, uma tabela especifica do banco de dados ou até mesmo uma coluna especifica de uma tabela do banco de dados, sendo possível criptografar os dados mesmo após a compactação dos dados para consultar esses dados se utiliza de uma tecnologia de chave de duas camadas, onde gera uma chave de criptografia para o banco de dados e outra chave de criptografia para a aplicação sendo que uma é gerenciada diretamente pelo banco de dados e a outra é armazenada e gerenciada fora do banco de dados, as chaves são criptografadas separadamente dos dados assim caso haja um ataque o será mínimo o dano, pois tendo somente as chaves não é possível fazer muita coisa, tal tecnologia permite muda as chaves sem precisar criptografar novamente os dados, sendo possível também configurar o tipo de acesso a esses dados, imaginemos um sistema de um Banco onde no cadastro do cliente possui informações como número de CPF, Cartão de credito, senha entre outras informações confidenciais que podem ser consultadas por funcionários do banco, com os níveis de acesso é possível já trazer os dados do banco de dados já mascarado, por exemplo o CPF ao invés de apresentar na tela 123.456.789-00 será exibindo somente ***.456.***-00, essa manipulação pode ser parcial ou completa, para aumentar a segurança existe um firewall de banco de dados  e um auditor de banco de dados onde juntos analisam e armazenam todo o tráfego de rede e atividades do sistema, analisando até mesmos os comando de SQL e determina com uma alta assertividade se deve ser permitido, registrado, alertado, substituído ou até mesmo bloqueado o comando recebido, já o controle de permissões de usuário é muito eficiente, controlando e acompanhando a todo tempo o que cada usuário está fazendo, caso haja uma atividade considerada suspeita essa atividade é automaticamente bloqueada, alguns comando SQL que venha muda o esquema do banco de dados onde crie, altere ou apague alguma estrutura do banco só pode ser realizada através da tela de gerenciamento do sistema, não permitindo que seja executada de outro local, caso chegue esse comando por dentro da aplicação ele será automaticamente bloqueado, ainda é possível também acompanhar em tempo real as avaliações e recomendações de melhoria na segurança do sistema, onde um serviço mostra problemas nas configurações de segurança e possíveis soluções, analisa também os usuários e seus direitos e até mostra a quantidade e tipo dos dados criptografados no banco de dados, e   todos esses recursos são implementados seguindo os níveis mais altos de segurança existente no mundo tudo isso sem produzir impactos de desempenho ao banco de dados, indo de encontro a um conceito antigo da informática que ressalta que a segurança é inversamente proporcional ao desempenho, quanto mais segurança é implementando mais recursos de maquina será necessário para atender a demanda \cite{OracleDefesa}.
	
	
Autorreparação: A cada dia que passa aumenta a necessidade das empresas de diminuir o tempo de inatividade e problemas que afetam o seu desempenho, pois os prejuízos ao decorrer de um ano podem ser milionários a depender da empresa. Um grande exemplo foi a empresa Amazon que em 2013 teve um downtime de 49 minutos que significou um prejuízo de 5 milhões de dólares em vendas, significando um déficit de 102 mil dólares por minuto. Isso sem contar os danos intangíveis como a perda de confiança dos clientes e uma má reputação da empresa, o que pode gerar danos muito maiores a longo prazo. Apesar desse número parecer dramático ele pode ficar pior, devemos lembrar de que as organizações incorrem em múltiplas interrupções por ano, representando várias horas ou até dias tempo de inatividade - que soma milhões de dólares anualmente. Esses tipos de falha podem vir de diversos eventos como falha de componentes, corrupção de dados ou algum erro humano, como aconteceu na empresa Alitalia em 2006 quando ofereceu voos do Canada para o Chipre por 39 dólares, até que o erro fosse descoberto 2 mil passagens já tinham sido vendidas gerando um prejuízo de 7,7 milhões de dólares \cite{Trapalhada}. Com o banco de dados autônomo é possível ter 99,995% de disponibilidade anual do banco de dados, isso significa 2,5 minutos ao mês ou 30 minutos ao ano, esse valor é um avanço muito grande, mas já existem estudos de soluções que prometem até quase 100% de disponibilidade, isso é possível pois o banco de dados autônomo monitora todos os serviços e estados do banco de dados, implementa de forma automática cópia de segurança do banco de dados onde ele realiza a cópia e validação das informações e depois distribui os arquivos em locais diferentes, analise do conteúdo das tabelas para descobrir possíveis corrupções de dados, replicação de banco de dados de forma automática em locais geograficamente diferentes com isso em caso de desastre a perda de dados será mínima ou quase zero, tudo isso usando machine learning assim aprendendo e corrigindo defeitos futuros. Alguns exemplos de serviços que a autorreparação pode fazer são: backup automatizado, disponibilidade automatizada de servidores, recuperação de desastres, manutenção e atualizações automatizadas e recuperação de erros humanos \cite{OracleAut}.


\section{Qual o novo papel do DBA}

O advento do banco de dados autônomo que minimiza a intervenção humana na sua administração, para muitos pode ser o fim da função de DBA, mas vemos isso como uma mudança das atribuições desse profissional, já que nesse novo ambiente o DBA não fará mais os backups, atualizações, replicações, restaurações e outras funções repetitivas e poderá assumir dentro da organização uma posição mais estratégica. Os profissionais poderão usar do seu conhecimento para gerar mais valor à empresa que trabalham, pois o tempo consumido anteriormente com tarefas manuais ficará disponível para pesquisas e inovações.


\section{Ferramentas que existem e o que fazem}

Presente no mercado há mais de 40 anos, a Oracle Corporation se consolidou como uma das maiores desenvolvedoras de banco de dados do mundo, além de ter também no seu portfólio de negócios atividades em infraestrutura de software e hardware.  Após anos de pesquisas em automação de banco de dados, automação da infraestrutura de banco de dados e desenvolvimento de nova tecnologia em infraestrutura em nuvem a Oracle desenvolveu um banco de dados totalmente autônomo.

Considerado o único banco de dados autônomo existente, \cite{Auricchio}, o Oracle Autonomous Database chegou para revolucionar o modo de como os bancos de dados são gerenciados, pois utiliza-se de inteligência artificial para Auto-Administração o que possibilita o mínimo de intervenção humana nos processos de administração do banco de dados. A Oracle Cloud Infraestructure,\cite{OCI}, é o serviço de infraestrutura de computação na nuvem da Oracle, onde o Oracle Autonomous Database está hospedado e é tarifado por hora, os usuários pagam somente pelo que usam.

O Oracle Autonomous Database foi desenvolvido para entregar diversos benefícios em três macro processos primários, que são: Auto-Gerenciamento, Auto-Proteção e Auto-Reparação. Cada macro processo é subdividido em diversos processos sendo todos eles realizados com o mínimo de intervenção humana.


\subsection{Auto-Gerenciamento}


Os recursos de Auto-Gerenciamento do Oracle Autonomous Database automatizam e simplificam muito o gerenciamento do banco de dados, que abrange 6 áreas principais. Como os recursos de Auto-Gerenciamento estão predominantemente associados à automação, alguns dos elementos de Auto-Reparação e Auto-Proteção que compõem a automação avançada também estão incluídos no recurso de Auto-Gerenciamento. De acordo com a Oracle, \cite{WPGestao}, os recursos de Auto-Gerenciamento incluem:

\begin{alineas}
\item Aprovisionamento: Quando um novo banco de dados é solicitado, o Oracle Autonomous Database Cloud provê um banco de dados escalável e altamente disponível, pré-criado, pré-configurado e pré-testado. As etapas que normalmente levariam horas ou dias para serem concluídas por um DBA são executadas em minutos. Isso inclui alocar e configurar o(s) servidor(es), máquina(s) virtual(is),  software e armazenamento necessário para implantar o banco de dados configurado no Oracle Cloud. Depois dessas etapas concluídas o banco de dados autônomo instala o banco de dados Oracle;

\item Segurança: Uma vez provisionado, o banco de dados precisa ser protegido. Por padrão, todos os dados no Oracle Auotonomous Database e todas as conexões com o banco de dados são totalmente criptografadas para proteger contra ataques e acesso não autorizados aos dados externos. As atualizações de segurança mais recentes são aplicadas on-line, geralmente trimestralmente, mas podem ocorrer em intervalos menores de tempo para solucionar vulnerabilidades de alto impacto;

\item Gerenciamento: Muito tempo de TI é gasto em manutenção em detrimento à inovação. Para simplificar a administração e reduzir ainda mais o risco de erro humano, a manutenção do banco de dados e da infraestrutura é automatizada. Nenhum administrador da empresa (ou membro da equipe da Oracle) ou contas do sistema operacional recebem privilégios de SYSDBA. A revogação desses privilégios garante que as credenciais do usuário não possam ser roubadas. Isso significa que o Oracle Autonomous Database deverá executar todas as operações de nível de SYSDBA e de nível de sistema operacional, incluindo toda a manutenção;

\item Proteção: O Oracle Autonomous Database faz seu backup na Oracle Cloud todas as noites e foi projetado para absorver automaticamente qualquer interrupção não planejada ou planejada de erros de componentes a desastres em toda a região. O processo de backup abrange uma ampla variedade de tecnologias e práticas recomendadas de alta disponibilidade e recuperação de desastres, do backup à replicação de dados com nenhuma perda de dados; 

\item Escalabilidade: A elasticidade é um dos principais motivos pelos quais as empresas desejam mudar para a nuvem, para rápida expansão e alocação de recursos e crescimento dos negócios. O Oracle Autonomous Database oferece a capacidade de escalar instantaneamente e automaticamente. CPUs individuais ou nós de computação e sistemas de armazenamento inteiros podem ser adicionados online. O Oracle Autonomous Database também aproveita a computação e o armazenamento sem servidor, o que permite que os clientes aumentem ou diminuam os recursos na medida certa promovendo assim um verdadeiro modelo de pagamento por uso;

\item Otimização: Essa é talvez a mais substantiva área do Auto-Gerenciamento do Oracle Autonomous Database. O Oracle Autonomous Database é otimizado para processamentos e volume de dados diferentes sem intervenção humana. Por exemplo, ele está atualmente disponível como um serviço de Data Warehouse Autônomo e um serviço de Processamento de Transações Autônomas. Ambos os serviços usam o Oracle Database (versão 18 e posterior), executados sobre a plataforma Exadata no Oracle Cloud, mas foram otimizados para cargas de trabalho muito diferentes, porém complementares. O Autonomous Data Warehouse foi projetado para análises rápidas e complexas, enquanto o Autonomous Transaction Processing foi projetado para executar um alto volume de transações simples.
\end{alineas}


\subsection{Auto-Proteção}


Os recursos de Auto-Proteção fornecem uma postura de segurança de linha de base, que se mostram superiores à maioria dos ambientes locais e são extensíveis o suficiente para atender aos requisitos de segurança mais rigorosos com facilidade. Conforme a Oracle, \cite{WPProtecao}, os recursos de Auto-Proteção incluem:

\begin{alineas} 
\item Criptografia para dados em movimento:  Cada serviço do Oracle Autonomous Database é configurado automaticamente para usar o padrão TLS 1.2 para criptografar dados em trânsito entre o serviço de banco de dados e clientes ou aplicativos. Os certificados de cliente e as informações de rede requeridas, são empacotados automaticamente para o solicitante quando o serviço for provisionado;

\item Criptografia para dados em repouso: Os dados no Oracle Autonomous Database são criptografados automaticamente usando o Oracle Transparent Data Encryption, disponível pela primeira vez com o Oracle 10g em 2004. A tecnologia Transparent Data Encryption vem sendo continuamente aprimorada desde a sua introdução. A Criptografia automática para dados em repouso e em movimento estão disponíveis apenas com o Oracle Cloud;

\item Separação automatizada de tarefas: O Oracle Autonomous elimina completamente o acesso direto ao nó do banco de dados e ao sistema de arquivos local. Um isolamento adicional entre os administradores e consumidores de serviços é fornecido pelo Oracle Database Vault, disponível pela primeira vez no Oracle Database 9i. Essa separação de tarefas, um importante diferencial do Oracle Cloud, não apenas reduz o risco de improbidade do administrador, como também elimina a capacidade dos administradores de serviço em exibir ou modificar dados armazenados no Oracle Autonomous Database. Assim como na tecnologia Oracle Transparent Data Encryption, o Oracle Database Vault vem sendo  aprimorado continuamente, com alguns novos recursos adicionados explicitamente para dar suporte ao Oracle Autonomous Database;

\item Auditoria de banco de dados configurada por padrão, personalizável para atender às necessidades dos administradores: O Oracle Autonomous Database é pré-configurado usando a tecnologia Oracle Unified Audit. Esse recurso inclui auditoria automatizada para atividades de usuário privilegiado, falhas de logon, políticas pré-configuradas opcionais para auditoria do CIS Benchmarks (Center for Internet Security), gerenciamento de contas, e muito mais;

\item Redução das oportunidades de erro humano: O erro humano desempenha um papel significativo em muitas violações de dados e é um dos vetores de ameaças mais difíceis de eliminar. O Oracle Autonomous Database minimiza as chances de erro humano, automatizando uma parte significativa da administração do banco de dados. As oportunidades de erro humano são reduzidas ainda mais ao restringir a gama de comandos que o administrador do Oracle Autonomous Database tem permissão para executar;

\item Patches, atualizações e manutenção automatizados: Uma das vantagens mais significativas do Oracle Autonomous Database é sua capacidade de aplicar automaticamente patches de segurança e atualizá-los sem tempo de inatividade. Grande parte dessa capacidade baseia-se em tecnologias conhecidas e bem testadas do Oracle Database, como o Real Application Clusters (para implantar patches RAC online) e a automação de processos de serviços em nuvem que evoluiu muito no Oracle Cloud.
\end{alineas}


\subsection{Auto-Reparo}


Integrada ao Oracle Autonomous Database é a MAA (Oracle Maximum Availability Architecture), tecnologia que evita (quando possível) e fornece recuperação automática de qualquer tipo de interrupção, eliminando o tempo de inatividade e a perda de dados para qualquer ambiente na nuvem, dos bancos de dados de teste aos ambientes de missão crítica. A maioria dos ambientes do Oracle Database já utiliza a tecnologia MAA como padrão de proteção e reparação dos dados.

Tecnologias autônomas diferenciadas são incluídas em cada nível subseqüente de proteção, começando com o backup como primeira linha de defesa e evoluindo para a replicação em tempo real para ambientes corporativos globais que não toleram tempo de inatividade ou perda de dados. Conforme a Oracle, \cite{WPReparo}, as tecnologias e recursos descritos abaixo são implementadas e ativadas automaticamente conforme necessário, com base no nível de serviço escolhido pela empresa no momento em que o serviço é iniciado.

\begin{alineas} 
\item Backup automatizado: O Oracle Autonomous Database executa backups automáticos noturnos para armazenamento de objetos no Oracle Cloud, onde os dados são protegidos contra interrupções do site com tripla redundância em vários data centers na nuvem e acessíveis 24 horas por dia, 7 dias por semana. As restaurações de backups na nuvem também são automatizadas;

\item Disponibilidade automatizada do servidor: O Oracle Real Application Clusters (RAC) é o padrão utilizado pela Oracle para alta disponibilidade do servidor de banco de dados. Ele foi projetado para permitir que um único banco de dados Oracle seja executado em vários nós em um cluster de alta disponibilidade para prover disponibilidade contínua e escalabilidade dinâmica. No caso de qualquer nó no cluster falhar, as sessões e os usuários do nó com falha são automaticamente e transparentemente movidos para outro nó no cluster, evitando tempo de inatividade e interrupção para os usuários. O Oracle RAC é a solução definitiva de alta disponibilidade para bancos de dados Oracle dentro de um único Domínio de Disponibilidade (data center) no Oracle Cloud;

\item Recuperação automatizada de desastres: Na inicialização, o Oracle Autonomous Database estabelece automaticamente uma configuração de expansão espelhada tripla em um data center em nuvem regional, com uma cópia em espera completa opcional em outra região para proteção contra desastres. Para recuperação de desastre entre cópias primárias e cópias em espera, o Data Guard fornece uma solução abrangente de replicação que elimina qualquer ponto único de falha para bancos de dados Oracle de missão crítica, desde falhas de componentes até falhas no banco de dados e interrupções no site. Ele evita a perda de dados e o tempo de inatividade de maneira simples e econômica, mantendo uma réplica física sincronizada (em espera) de um banco de dados de produção (primário) no Oracle Cloud. O Oracle Data Guard automaticamente aciona um banco de dados em espera se o banco de dados primário ficar indisponível por qualquer motivo. As conexões do cliente fazem uma mudança rápida e automática para o modo de espera atualizado e o serviço é retomado;

\item Manutenção e atualizações automatizadas: Muito tempo de TI é gasto em manutenção. O Banco de Dados Autônomo oferece muitos recursos para eliminar o tempo de inatividade planejado, o que é cada vez mais problemático para organizações em crescimento que precisam oferecer suporte a mais servidores, bancos de dados, aplicativos e consequentemente, gastam mais tempo com atualizações, patches e migrações;

\item Recuperação automatizada de erros humanos: Muitas das interrupções não planejadas ocorrem devido a problemas de pessoas e processos. O Flashback Database é um recurso do Oracle Autonomous Database que automatiza a recuperação de erros humanos. Foi chamado de "rewind button" que permite que o banco de dados seja revertido e visualizado em um momento anterior para investigação de erros e recuperação rápida;

\item Otimização da Infraestrutura do Oracle Database Cloud: O Oracle Autonomous Database é executado na infraestrutura do Oracle Database Exadata Cloud, que é baseada no hardware da Exadata Database Machine. A infraestrutura em nuvem baseada no Exadata inclui todos os recursos do Oracle MAA e fornece recursos de computação, rede integrados e otimizados para o Oracle Database.
\end{alineas} 


Os bancos de dados autônomos já são realidade para algumas grandes organizações, e em pouco tempo serão realidade nas médias e pequenas empresas, onde os recursos de Auto-Gerenciamento, Auto-Proteção e Auto-Reparo serão fatores decisivos para a substituição das soluções de banco de dados dependente de gerenciamento humano.




