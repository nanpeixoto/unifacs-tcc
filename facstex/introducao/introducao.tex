\chapter{Introdução}

A inteligência artificial não é algo novo. Essa pesquisa acontece desde os anos 50, onde se estimava-se que máquinas poderiam resolver problemas simples e realizar tarefas autônomas.


Atualmente, com o mundo globalizado e com a extrema competitividade que existe no mundo empresarial, é crescente a exigência que as empresas armazenem e gerenciem uma enorme e variada quantidade de dados, além disso, é imprescindível a utilização inteligente desses dados valiosos para auxiliar na toma de decisões. É diário o desafio de receber esses dados e mantê-los disponível pelo maior número de tempo possível. 


Para atender esse desafio, surgiu a figura do DBA, que traduzindo para o português significa Administrador de dados, ele é o responsável pela instalação, configuração, atualização, otimização, segurança, monitoramento e gerenciamento de um banco de dados (DB), ou seja, qualquer processo relacionado a um banco de dados que demande uma decisão gerencial ágil e assertiva, além de fazer backup e aprimoramentos de performance.


De acordo com o blog \cite{OracleBlog}, a IDC, que é uma empresa líder em inteligência de mercado e consultoria nas indústrias de tecnologia da informação, telecomunicações e mercados de consumo em massa de tecnologia, prevê que a quantidade de informações geradas cresça a um fator 10, duplicando de tamanho a cada 02 anos.


No meio desse cenário surge a aplicação da inteligência artificial para auxiliar o DBA no monitoramento e gerenciamento desse conjunto grande e variado de dados.