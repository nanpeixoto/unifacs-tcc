\chapter{Introdução}

A inteligência artificial não é algo novo. Essa pesquisa acontece desde os anos 50, onde se estimava-se que máquinas poderiam resolver problemas simples e realizar tarefas autônomas.


Atualmente, com o mundo globalizado e com a extrema competitividade que existe no mundo empresarial, é crescente a exigência que as empresas armazenem e gerenciem uma enorme e variada quantidade de dados. De acordo com o blog (PINA, 2018), a IDC, que é uma empresa líder em inteligência de mercado e consultoria nas indústrias de tecnologia da informação, telecomunicações e mercados de consumo em massa de tecnologia, prevê que a quantidade de informações geradas cresça a um fator 10, duplicando de tamanho a cada 02 anos.


Para atender esse desafio, surgiu a figura do DBA, que traduzindo para o português significa Administrador de dados, ele é o responsável pela criação, manutenção e resolução de possíveis problemas que podem acontecer. Por isso, buscar sempre uma performance eficiente com uma modelagem coerente, garantir que a rotina de backup esteja sendo executada e diariamente monitorar as bases de dados são extremamente importante para o pleno funcionamento.

No meio desse cenário surge a aplicação da inteligência artificial para auxiliar o DBA automatizando muitas rotinas, evitando falhas humanas e ao mesmo tempo deixando o profissional mais focado em atividades estratégicas da empresa.